\documentclass[twocolumn]{article}

\usepackage{cameronsBasePackage}

\title{A test document}
\author{Cameron Roe, in all likelihood}
\dueDate{\today}
\classname{AME 100}

\begin{document}

\maketitle

\section{I'm a section}
Hello, this is a test. 

\problem{1}

For any gas, 

\begin{equation}
    a = \left[ \left(\frac{\partial p}{\partial \rho} \right)_s \right]^{\frac{1}{2}} = \frac{1}{\sqrt{\rho \beta_s}}
\end{equation}

For a calorically perfect gas undergoing an isentropic process, 

\begin{equation}
\begin{split}
    \left(\frac{p_2(T,\rho))}{p_1}\right)^\frac{\gamma - 1}{\gamma} &= \left( \frac{\rho_2(T)}{\rho_1} \right)^{\gamma - 1} \\
    \vdots& \\ 
    \frac{p_2(T,\rho)}{p_1} &= \left(\frac{\rho_2(T)}{\rho_1} \right)^\gamma
    \end{split}
\end{equation}

Assume infinitesimally small changes in pressure and density such that $p_2 = p + dp$ and $\rho_2 = \rho + d\rho$. Therefore, 

\begin{equation}
    1 + \frac{dp}{p} = \left(1 + \frac{d\rho}{\rho}\right)^\gamma
\end{equation}

Using the first term of the Taylor series expansion for binomials and assuming that $\frac{d\rho}{\rho} \ll 1$, we get

\begin{equation}
    \frac{dp}{p} = \gamma \frac{d\rho}{\rho}
\end{equation}

Therefore, we can derive the isentropic compressibility of a calorically perfect gas to be

\begin{equation}
    \beta_s \equiv \frac{1}{\rho} \left(  \frac{\partial \rho}{\partial p}\right)_s =\left( \frac{1}{\rho}\right) \frac{1}{\gamma} \frac{\rho}{p} = \frac{1}{\gamma p} \label{eq:isentropicCompressibility}
\end{equation}

For an ideal gas, the isothermal compressibility is 

\begin{equation}
    \beta_T = \frac{1}{\rho} \left(\frac{\partial \rho}{\partial p} \right)_T = \frac{1}{p}
\end{equation}

Therefore, $\beta_T = \gamma- \beta_s$ for a calorically perfect gas. Thus, the speed of sound of a calorically perfect gas can be expressed using the isothermal compressibility as 

\begin{equation}
    \boxed{a = \sqrt{\frac{\gamma}{\rho \beta_T}}}
    \end{equation}

\pagebreak

\pagebreak
\discussion{1}
This should be on a different page or column.

\section{A graph of a function}

\newcommand{\E}{\mathrm{e}} 
\begin{figure}[h]
	\centering
	\begin{tikzpicture}
	\begin{axis}[
		my axis style,
		width=.5\textwidth,
		height=.5\textwidth,
		legend entries={
			$y = x\E^{-x}$,
			$y = 2x\E^{-x}$,
			$y = 0.5x\E^{-x}$
		},
		legend pos=north east
	]
	
	\addplot[
		domain=0:5,
		thick,
		->
	]
	{x*exp(-x)};

	\addplot[
		domain=0:5,
		thick,
		red,
		dashed,
		->
	]
	{2*x*exp(-x)};

	\addplot[
		domain=0:5,
		thick,
		blue,
		dashdotted,
		->
	]
	{.5*x*exp(-x)};
	
	\fill[
		black
	]
	(1,.36788) circle (2pt) node[above right] {\footnotesize $(1, 1/\E)$};
	
	\end{axis}
	\end{tikzpicture}
	\caption{}
	\label{fig:my-awesome-graph}
\end{figure}

\pagebreak
\appendix
\section{Code listing}

\begin{minted}[
linenos,
fontsize=\footnotesize,
fontfamily=tt,
style=sas,
mathescape=true
]{matlab}
function [fluxArray,indexArray] = getFlux(solution)
% GETFLUX compute 2nd order approximation of the boundary flux for each solution poin
arguments
    solution (:,:) {mustBeSquareMatrix} = rand(25);
end

%find the grid spacing
delta = 1 / (length(solution)-1); 

%initialize fluxArray
fluxArray = zeros(length(solution),length(enumeration('FluxDirections')));

% assign values column by column by iterating through enumerated directions
col = 1;

%resolve the solution edges into the corresponding indices
leftEdge = [(1 : 25)',ones(25,1)];
leftEdge = flip(leftEdge);

%tryhard LaTeX comment: $E = mc^2$

rightEdge = [(1 : 25)', 25*ones(25,1)];
rightEdge = flip(rightEdge);

topEdge = [ones(25,1),(1:25)'];

bottomEdge = [25*ones(25,1), (1:25)'];

%compute fluxes
for fluxDirection = enumeration('FluxDirections')'
    %disp(fluxDirection)
    if fluxDirection == FluxDirections.LEFT
        edge = leftEdge;
    elseif fluxDirection == FluxDirections.RIGHT
        edge = rightEdge;
    elseif fluxDirection == FluxDirections.BOTTOM
        edge = bottomEdge;
    elseif fluxDirection == FluxDirections.TOP
       edge = topEdge;
    else
        error(strcat('Unable to resolve flux direction: ',char(fluxDirection)));
    end
    
   for idx = 1 : length(solution)
       fluxArray(idx,col) = fluxFromDifference(edge(idx,:),solution,fluxDirection,delta);
       indexArray{idx,col} = edge(idx,:);
   end
   col = col + 1;
end

end

\end{minted}


\end{document}